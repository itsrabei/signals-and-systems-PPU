\documentclass{article}
\usepackage{amsmath}
\usepackage{amssymb}
\usepackage{graphicx}
\usepackage{tikz}
\usepackage{pgfplots}
\usepackage{float}
\usepackage{subcaption}
\usepackage{geometry}

\geometry{a4paper, margin=1in}

% Define example environment
\newenvironment{example}[1]{
    \begin{trivlist}
    \item[\textbf{Example:}] #1
    \vspace{0.5em}
}{
    \end{trivlist}
    \vspace{1em}
}

\pgfplotsset{compat=1.18}
\usetikzlibrary{patterns,decorations.pathreplacing,shapes.geometric,arrows.meta,positioning}

\title{Lecture 3 Examples}
\author{Signals and Systems Course}
\date{}

\begin{document}

\maketitle

\begin{example}[1. Determining if a System has Memory]
\textbf{Problem:}
Determine whether the system $y[n] = (2x[n] - x[n]^2)^2$ is memoryless.

\begin{figure}[H]
    \centering
    % In the preamble:
% \usetikzlibrary{positioning,arrows.meta}

\begin{tikzpicture}[
	auto,
	node distance=1.5cm and 2cm, % vertical and horizontal distances via positioning
	>=Stealth, % requires arrows.meta
	block/.style={draw, rectangle, minimum height=3em, minimum width=4em},
	adder/.style={draw, circle, inner sep=0pt, minimum size=0.7cm}
	]
	
	% --- NODES ---
	\node (input) {$x[n]$};
	\coordinate[right=1cm of input] (branch);
	\node[block, right=of branch] (gain) {$2$};
	\node[block, below=of gain] (sq1) {$(\cdot)^2$};
	
	% Summing junction placed to the right of the upper path
	\node[adder, right=2cm of gain] (sum) {};
	
	\node[block, right=1.5cm of sum] (sq2) {$(\cdot)^2$};
	\node[right=1.5cm of sq2] (output) {$y[n]$};
	
	% --- CONNECTIONS ---
	\draw[->] (input) -- (branch);
	\draw[->] (branch) |- (gain.west);
	\draw[->] (branch) |- (sq1.west);
	
	% Feed the adder cleanly
	\draw[->] (gain.east) -- (sum.west);
	\draw[->] (sq1.east) -| (sum.south);
	
	% Output path
	\draw[->] (sum.east) -- (sq2.west);
	\draw[->] (sq2.east) -- (output);
	
	% --- LABELS ---
	% Place + at the upper input and - at the lower input of the adder
	\node[above=2pt of sum.north] {$+$};
	\node[left=2pt of sum.south] {$-$};
	
	% Optional: system equation
	\node[below=1.5cm of sq1, text width=8cm, align=center] (equation)
	{System Equation: $y[n] = \left(2x[n] - (x[n])^2\right)^2$};
	
\end{tikzpicture}

    \caption{Block diagram representation of the system.}
    \label{fig:block_diagram}
\end{figure}

\textbf{Solution:}

The output $y[n]$ depends only on the current input $x[n]$ at time $n$. There is no dependence on past or future values.

\textbf{Answer:} The system is memoryless.
\end{example}

\vspace{0.5em}
\hrule
\vspace{0.5em}

\begin{example}[2. System Properties - Memory and Invertibility]
\textbf{Problem:}
Determine which systems have memory and which are invertible:
\begin{enumerate}
    \item $y(t) = x(t) + 1$ (continuous-time)
    \item $y[n] = x[n-1]$ (discrete-time)
    \item $y(t) = x^2(t)$
\end{enumerate}

\textbf{Solution:}
\begin{itemize}
    \item System 1: Memoryless (depends only on current input)
    \item System 2: Has memory (depends on past input)
    \item System 3: Not invertible ($x_1 = 2$ and $x_2 = -2$ both give $y = 4$)
\end{itemize}

\textbf{Answer:} System 1 is memoryless; System 2 has memory; System 3 is not invertible.
\end{example}

\vspace{0.5em}
\hrule
\vspace{0.5em}

\begin{example}[3. System Invertibility Analysis]
\textbf{Problem:}
Determine the invertibility of:
\begin{enumerate}
    \item $y[n] = 2x[n]$ (discrete-time)
    \item $y(t) = \int_{-\infty}^{t} x(\tau) d\tau$ (continuous-time integrator)
    \item $y[n] = x^2[n]$ (discrete-time)
\end{enumerate}

\textbf{Solution:}
\begin{itemize}
    \item System 1: Invertible ($x[n] = y[n]/2$)
    \item System 2: Invertible ($x(t) = dy(t)/dt$)
    \item System 3: Not invertible ($x_1 = 3$ and $x_2 = -3$ both give $y = 9$)
\end{itemize}

\textbf{Answer:} Systems 1 and 2 are invertible; System 3 is not invertible.
\end{example}

\vspace{0.5em}
\hrule
\vspace{0.5em}

\begin{example}[4. System Causality Analysis]
\textbf{Problem:}
Determine the causality of:
\begin{enumerate}
    \item $y[n] = x[n] + x[n-1]$ (discrete-time)
    \item $y[n] = x[n+1]$ (discrete-time)
    \item $y(t) = x(t-2)$ (continuous-time)
    \item $y(t) = x(t+1)$ (continuous-time)
\end{enumerate}

\textbf{Solution:}
\begin{itemize}
    \item System 1: Causal (current and past inputs)
    \item System 2: Non-causal (future input)
    \item System 3: Causal (past input)
    \item System 4: Non-causal (future input)
\end{itemize}

\textbf{Answer:} Systems 1 and 3 are causal; Systems 2 and 4 are non-causal.
\end{example}

\vspace{0.5em}
\hrule
\vspace{0.5em}

\begin{example}[5. Testing for Time-Invariance]
\textbf{Problem:}
Determine if $y(t) = \sin[x(t)]$ is time-invariant.

\textbf{Solution:}
Test: If $x_1(t) \to y_1(t) = \sin[x_1(t)]$, then $x_1(t-t_0) \to y_2(t) = \sin[x_1(t-t_0)]$
Compare: $y_1(t-t_0) = \sin[x_1(t-t_0)] = y_2(t)$

\textbf{Answer:} Time-invariant (sine operates on instantaneous values)
\end{example}

\vspace{0.5em}
\hrule
\vspace{0.5em}

\begin{example}[6. Testing a System for Time-Invariance]
\textbf{Problem:}
Determine if $y[n] = n x[n]$ is time-invariant.

\textbf{Solution:}
Test: If $x_1[n] \to y_1[n] = n x_1[n]$, then $x_1[n-n_0] \to y_2[n] = n x_1[n-n_0]$
Compare: $y_1[n-n_0] = (n-n_0) x_1[n-n_0] \neq y_2[n] = n x_1[n-n_0]$

\textbf{Answer:} Time-varying (scaling factor depends on time index $n$)
\end{example}

\vspace{0.5em}
\hrule
\vspace{0.5em}

\begin{example}[7. Testing a Time-Scaling System for Time-Invariance]
\textbf{Problem:}
Determine if $y[n] = x[2n]$ (decimator) is time-invariant.

\textbf{Solution:}
Test: If $x_1[n] \to y_1[n] = x_1[2n]$, then $x_1[n-n_0] \to y_2[n] = x_1[2n-n_0]$
Compare: $y_1[n-n_0] = x_1[2n-2n_0] \neq y_2[n] = x_1[2n-n_0]$

\textbf{Answer:} Time-varying (decimation samples at specific indices that change with shift)
\end{example}

\vspace{0.5em}
\hrule
\vspace{0.5em}

\begin{example}[8. Proving Linearity for an Ideal Sampling System]
\textbf{Problem:}
Determine if $y(t) = \sum_{n=-\infty}^{\infty} x(t) \delta(t - nT)$ is linear.

\textbf{Solution:}
\textbf{Homogeneity:} $\mathcal{T}\{\alpha x(t)\} = \sum_{n=-\infty}^{\infty} \alpha x(t) \delta(t - nT) = \alpha \sum_{n=-\infty}^{\infty} x(t) \delta(t - nT) = \alpha y(t)$

\textbf{Additivity:} $\mathcal{T}\{x_1(t) + x_2(t)\} = \sum_{n=-\infty}^{\infty} [x_1(t) + x_2(t)] \delta(t - nT) = y_1(t) + y_2(t)$

\textbf{Answer:} Linear (satisfies both homogeneity and additivity)
\end{example}

\vspace{0.5em}
\hrule
\vspace{0.5em}

\begin{example}[9. Testing a System for Linearity]
\textbf{Problem:}
Determine if $y[n] = |x[n]|$ is linear.

\textbf{Solution:}
\textbf{Additivity:} $|x_1[n] + x_2[n]| \neq |x_1[n]| + |x_2[n]|$ (e.g., $|1 + (-1)| = 0 \neq 2 = |1| + |-1|$)

\textbf{Homogeneity:} $|\alpha x[n]| = |\alpha| \cdot |x[n]| \neq \alpha |x[n]|$ when $\alpha < 0$

\textbf{Answer:} Non-linear (fails both additivity and homogeneity)
\end{example}

\vspace{0.5em}
\hrule
\vspace{0.5em}

\begin{example}[10. Systems Exhibiting Partial Linearity Properties]
\textbf{Problem:}
Determine linearity of:
\begin{enumerate}
    \item $y[n] = \frac{x[n]x[n-1]}{x[n+1]}$
    \item $y[n] = \text{Re}\{x[n]\}$
\end{enumerate}

\textbf{Solution:}
\textbf{System 1:} Homogeneous ($\alpha$ cancels out) but not additive (nonlinear operations)
\textbf{System 2:} Additive but not homogeneous (fails for complex $\alpha$)

\textbf{Answer:} Both non-linear (need both additivity and homogeneity)
\end{example}

\vspace{0.5em}
\hrule
\vspace{0.5em}

\begin{example}[11. Stability Analysis of a Continuous-Time Integrator]
\textbf{Problem:}
Determine if $y(t) = \int_{-\infty}^{t} x(\tau) \,d\tau$ is stable.

\textbf{Solution:}
\textbf{Impulse response:} $h(t) = u(t)$ (apply $\delta(t)$ as input)

\textbf{Stability test:} $\int_{-\infty}^{\infty} |h(t)| \,dt = \int_{0}^{\infty} 1 \,dt = \infty$

\textbf{Answer:} Unstable (impulse response not absolutely integrable)
\end{example}

\end{document}